\chapter{Input and Output files format}

In program \emph{Surcos}, the input and output files are stored in a single common folder whose name
can be chosen by the user. However, the input and output files are plain ASCII files with a fixed name
that follows the structure detailed in next sections


\section{Input files}

\subsection{The geometry file: field.in}

The geometry file is always named:
\begin{itemize}
\item Case\_folder$\backslash$field.in
\end{itemize}

This file contains the following set of numbers:\\
$o$ $n$ $s$\\
$x_1$ $y_1$ $z_1$\\
$x_2$ $y_2$ $z_2$\\
$x_3$ $y_3$ $z_3$\\
$x_4$ $y_4$ $z_4$\\
$b_b$ $z_b$ $H_b$ $D_b$ $R_b$ $\epsilon_b$ $r_b$ $K_b$ $a_b$ $i_b$ $d_b$
	$h_b$ $Q_b$ $c_b$\\
$b_i$ $z_i$ $H_i$ $D_i$ $R_i$ $\epsilon_i$ $r_i$ $K_i$ $a_i$ $i_i$ $d_i$
	$h_i$ $Q_i$ $c_i$\\
$b_c$ $z_c$ $H_c$ $D_c$ $R_c$ $\epsilon_c$ $r_c$ $K_c$ $a_c$ $i_c$ $d_c$
	$h_c$ $Q_c$ $c_c$\\
with:
\begin{description}
\item[$o$] 1 if there is recirculation furrow, 0 otherwise.
\item[$n$] Number of irrigation furrows. 0 if there is a single furrow. 
           In that case, the flow in the distribution furrow is simulated.
\item[$s$] Fertilizer solubility.
\item[$x_j$, $y_j$, $z_j$] Spatial coordinates of vertex $j$ defining one corner of the 
	furrow network plot.  
\item[$b_k$] Base width in furrow $k$.
\item[$z_k$] Lateral slope in furrow $k$.
\item[$H_k$] Maximum depth in furrow $k$.
\item[$D_k$] Furrow separation at furrow $k$.
\item[$R_k$] Water retention capacity in furrow $k$.
\item[$\epsilon_k$] Burguete aerodynamic roughness coefficient ($\epsilon>0$) or Gauckler-Manning 
            roughness coefficient($\epsilon=0$) at furrow $k$.
\item[$r_k$] Gauckler-Manning number ($\epsilon=0$) or Burguete roughness characteristic length
            ($\epsilon>0$) at furrow $k$.
\item[$K_k$] Kostiakov infiltration coefficient at furrow $k$.
\item[$a_k$] Kostiakov infiltration power at furrow $k$.
\item[$i_k$] Infiltration saturated rate at furrow $k$.
\item[$d_k$] Diffusion coefficient at furrow $k$.
\item[$h_k$] Initial surface water depth at furrow $k$.
\item[$Q_k$] Initial surface water discharge at furrow $k$.
\item[$c_k$] Initial fertilizer concentration at furrow $k$.
\end{description}
The subindex label $k$ represents: $b$ for the distribution furrow, $i$ for all the irrigation furrows 
(they are all assumed to share the same geometry properties) and $c$ for the recirculation furrow.

\subsection{Inlet water and fertilizer file: input.in}

Water and fertilizer inlet is defined in a file named:
\begin{itemize}
\item Case\_folder$\backslash$input.in
\end{itemize}

The following set of data must be provided:\\
$n_w$ $n_s$\\
$x_1$ $y_1$ $I_1$ $F_1$ $q_1$\\
$\cdots$\\
$x_{n_w}$ $y_{n_w}$ $I_{n_w}$ $F_{n_w}$ $q_{n_w}$\\
$x_1$ $y_1$ $I_1$ $F_1$ $q_1$\\
$\cdots$\\
$x_{n_s}$ $y_{n_s}$ $I_{n_s}$ $F_{n_s}$ $q_{n_s}$\\
with:
\begin{description}
\item[$n_w$, $n_s$] the number of water and fertilizer inlet points respectively,
\item[$x_i$] $x$ coordinate of the $i$ inlet point,
\item[$y_i$] $y$ coordinate of the $i$ inlet point,
\item[$I_i$] initial time at the $i$ inlet point,
\item[$F_i$] final time at the $i$ inlet point,
\item[$q_i$] Inlet discharge at the $i$ inlet point, in $m^3/s$ if it is a water inlet or in 
	 $kg/s$ when it is a fertilizer inlet.
\end{description}
Note that first the $n_w$ water inlet points must be defined and then the $n_s$ fertilizer inlet points.

\subsection{Simulation time file: times.in\label{SecTiempos}}

The relevant time data in the simulation are defined in file:
\begin{itemize}
\item Case\_folder$\backslash$times.in
\end{itemize}

This file contains the following information:\\
$t_f$ $cfl$ $t_m$\\
where:
\begin{description}
\item[$t_f$] is the total simulation time,
\item[$cfl$] is the CFL number controlling the time step,
\item[$t_m$] is the time interval to write intermediate numerical results in a file.
\end{description}

\subsection{Grid file: mesh.in}

The number of grid computational grid cells is defined in file:
\begin{itemize}
\item Case\_folder$\backslash$mesh.in
\end{itemize}

This file contains:\\
$n_d$ $n_i$\\
with:
\begin{description}
\item[$n_d$] number of cells inserted in the distribution/recirculation furrows between each pair of irrigation furrows,
\item[$n_i$] number of cells in every irrigation furrow.
\end{description}

\subsection{Probes file: probe.in}

The measurement points or probes are specified in the file:
\begin{itemize}
\item Case\_folder$\backslash$probe.in
\end{itemize}

This file contains the following data:\\
$n_p$\\
$x_1$ $y_1$\\
$\cdots$\\
$x_{n_p}$ $y_{n_p}$\\
with:
\begin{description}
\item[$n_p$] total number of probes,
\item[$x_i$] $x$ coordinate of the location of the $i$ probe,
\item[$y_i$] $y$ coordinate of the location of the $i$ probe.
\end{description}

\subsection{Model file: model.in}

The models used are defined in the file:
\begin{itemize}
\item Case\_folder$\backslash$model.in
\end{itemize}

This file contains the following data:\\
$m_f$ $m_i$ $m_u$ $m_a$\\
with:
\begin{description}
\item[$m_f$] 1 if there is fertilizer transport, 0 otherwise.
\item[$m_i$] 1 if there is infiltration, 0 otherwise.
\item[$m_u$] 1 if there is turbulent diffusion, 0 otherwise.
\item[$m_a$] 1 if the Boussinesq convection model is used, 0 if the simple model is used.
\end{description}

\section{Results files}

In program \emph{Surcos} the results are gathered in files according to the following names: 
\begin{description}
\item[00b] Distribution furrow.
\item[00c] Recirculation furrow.
\item[xxx] Irrigation furrow, with $xxx$ representing the furrow number with 3
	digits. The furrow numbering goes from 0 to $n$-1 with $n$ the total number of irrigation furrows.
\end{description}

The results files are written in the same folder where the input data files are. The program generates
3 different results files, all of them in ASCII format.

\subsection{Longitudinal profile files (xxx-yyy.out)}

\emph{Surcos} generates a file to store the longitudinal profile of every variable in every furrow for every
intermediate time as defined in section~\ref{SecTiempos}). The name of these files is as follows:

\begin{itemize}
\item $xxx-yyy.out$
\end{itemize}
where $xxx$ represents the furrows name with the above described codification and $yyy$ represents
the intermediate time step with 3 digits.

The profiles are provided using files with 12 columns in the form:\\
\begin{tabular}{cccccccccccc}
$x_1$& $h_1$& $A_1$& $Q_1$& ${z_s}_1$& ${z_f}_1$& $-{A_i}_1$& $c_1$&
	$-{A_{ci}}_1$& $-{A_p}_1$& $-{A_{cp}}_1$& $\beta_1$\\
&&&&&&$\cdots$\\
$x_n$& $h_n$& $A_n$& $Q_n$& ${z_s}_n$& ${z_f}_n$& $-{A_i}_n$& $c_n$&
	$-{A_{ci}}_n$& $-{A_p}_n$& $-{A_{cp}}_n$& $\beta_n$
\end{tabular}\\\\
with:
\begin{description}
\item[$x_i$] the longitudinal coordinate of the $i$ grid point,
\item[$h_i$] the surface water depth of the $i$ grid point,
\item[$A_i$] the surface wetted cross section of the $i$ grid point,
\item[$Q_i$] the surface water discharge of the $i$ grid point,
\item[${z_s}_i$] the water surface elevation of the $i$ grid point,
\item[${z_f}_i$] the bed level elevation of the $i$ grid point,
\item[$-{A_i}_i$] the infiltrated water volume (negative) per unit furroe length of the $i$ grid point,
\item[$c_i$] the surface fertilizer concentration of the $i$ grid point,
\item[$-{A_{ci}}_i$] the mass of fertilizer infiltrated (negative) per unit furrow length of the $i$ grid point,
\item[$-{A_p}_i$] the volume of percolated water per unit furrow length of the $i$ grid point,
\item[$-{A_{cp}}_i$] the mass of percolated fertilizer per unit furrow length of the $i$ grid point,
\item[$\beta_i$] the Boussinesq convection coefficient of the $i$ grid point, 
\item[$n$] the number of grid points in the furrow.
\end{description}

\subsection{Advance and recession times files (xxx.out)}

\emph{Surcos} generates files with information concerning the advance and recession times in every furrow. 
The names of the files are in the following form: 
\begin{itemize}
\item $xxx.out$
\end{itemize}
where $xxx$ represents the name of the furrow with the above described codification. 

These files contain 3 columns in the form:\\
\begin{tabular}{ccc}
$x_1$& ${t_a}_1$& ${t_r}_1$\\
&$\cdots$\\
$x_n$& ${t_a}_n$& ${t_r}_n$\\
\end{tabular}\\\\
with:
\begin{description}
\item[$x_i$] the longitudinual aoordinate of the $i$ grid cell,
\item[${t_a}_i$] the advance time for the $i$ grid cell,
\item[${t_r}_i$] the recession time for the $i$ grid cell,
\item[$n$] the number of grid points in the furrow.
\end{description}

\subsection{Probe files (probes.out)}

The files where the probe information is stored are called \emph{probes.out}. This file is written with the following format:\\
\begin{tabular}{cccccc}
$t_0$ & $h_{1,0}$ & $c_{1,0}$ & $\cdots$ & $h_{n_p,0}$ & $c_{n_p,0}$\\
&&&$\vdots$\\
$t_{n_t}$ & $h_{1,n_t}$ & $c_{1,n_t}$ & $\cdots$ & $h_{n_p,n_t}$ & $c_{n_p,n_t}$
\end{tabular}\\\\
with:
\begin{description}
\item[$t_j$] the $j$ time level, with 0 the initial time,
\item[$h_{i,j}$] the surface water depth measured in probe $i$ at time $j$,
\item[$c_{i,j}$] the surface fertilizer concentration measured in probe $i$ at time $j$,
\item[$n_p$] the total number of probes,
\item[$n_t$] the total number of time steps.
\end{description}
