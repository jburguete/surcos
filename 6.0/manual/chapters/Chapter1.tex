\chapter*{Warning!}

To execute the program \emph{surcos} on Microsoft Windows operative system init
the application on the subdirectory \emph{win64/bin/winsurcos.exe}.

\begin{itemize}
\item Real numbers are represented according to the international standard,
overriding locale settings, separating the integer and decimal parts by ''.''.
\item All units are in the International System.
\end{itemize}

The program has been tested in the following operative systems of Microsoft:
\begin{itemize}
\item Windows 10 64 bits.
\end{itemize}

The language activation is perfomed through the regional configuration in the 
operative system. Version 6.0 is offered in English, Spanish, French and
Italian. 

Some problems with graphic representation have been reported due to OpenGL 
configuration failure in the graphical card driver. In that case, the latest
graphic card driver should be installed.

Windows 10 is a trademarks of Microsoft.

\setcounter{page}{1}

\chapter{Install instructions}

\section{Download}

Program \emph{surcos} can be freely downloaded from:
\begin{itemize}
\item \url{http://digital.csic.es/handle/10261/75830}
\end{itemize}
The latest program \emph{surcos} source code can be freely downloaded, under
a BSD type licence, from:
\begin{itemize}
\item \url{https://github.com/jburguete/surcos}
\end{itemize}

\section{Source code files}

\begin{description}
\item[configure.ac]: configure generator.
\item[Makefile.in]: Makefile generator.
\item[TODO]: list of tasks TO DO.
\item[src/config.h.in]: config header generator.
\item[src/*.h]: C-header files.
\item[src/*.c]: C-source files.
\item[*.png]: diagram and logo files.
\item[graph/*.tex]: LaTeX files to generate the diagrams.
\item[Doxyfile]: configuration file to generate doxygen documentation.
\item[locale/es/LC\_MESSAGES/*.po]: spanish language files.
\item[locale/fr/LC\_MESSAGES/*.po]: french language files.
\item[locale/it/LC\_MESSAGES/*.po]: italian language files.
\item[examples/*.json]: example input files.
\item[check-errors/*.json]: input files to check error messages.
\item[manual/*]: manual files.
\end{description}

\section{Instalation of the binary program in Microsoft Windows}

To install \emph{surcos} it must only be decompressed in the desired folder.
It is recommended to avoid blank spaces and symbols in the names of the folder. 

\subsection{Program folders for Microsoft Windows}

The program contains the following folders:
\begin{description}
\item[win64/bin]
\item Folder with the executable file, the libraries executable files and the
diagram files.
\item[win64/etc]
\item Configuration folder for the libraries.
\item[win64/lib]
\item It contains some binary files of the libraries.
\item[win64/share]
\item Contains the language files for Windows 32 bits.
\item[examples]
\item Example files.
\item[src]
\item Source code of \emph{surcos} and source code of the libraries used.
\end{description}

\section{Source code compilation}

\subsection{Required library and tools}

The source code is written in C and has been compiled using the following free
libraries and tools:

\begin{itemize}

\item Mandatory:
\begin{description}
\item[gcc]\url{https://gcc.gnu.org} or
\item[clang]\url{https://clang.llvm.org}
\item to compile the source code.
\item[make]\url{https://www.gnu.org/software/make}
\item to build the executable file.
\item[autoconf]\url{https://www.gnu.org/software/autoconf}
\item to generate the Makefile in different operative systems.
\item[automake]\url{https://www.gnu.org/software/automake}
\item to check the operative system.
\item[pkg-config]\url{https://www.freedesktop.org/wiki/Software/pkg-config}
\item to find the libraries to compile.
\item[glib]\url{https://developer.gnome.org/glib}
\item extended utilities of C to work with data, lists, mapped files, regular
expressions, using multicores in shared memory machines, ...
\item[json-glib]\url{https://gitlab.gnome.org/GNOME/json-glib}
\item to deal with JSON files.
\item[gettext]\url{https://www.gnu.org/software/gettext}
\item to work with different international locales and languages.
\item[jb]\url{https://github.com/jburguete/jb.git}
\item tools library of J. Burguete.
\end{description}

\item Optional to get the processor properties:
\begin{description}
\item[libgtop]\url{https://github.com/GNOME/libgtop}
\item to get the processors number.
\end{description}

\item Optional: required only to build the GUI program:
\begin{description}
\item[png]\url{http://libpng.sourceforge.net}
\item to work with PNG files.
\item[gtk]\url{https://www.gtk.org}
\item to create the interactive GUI tool.
\item[glew]\url{https://glew.sourceforge.net}
\item high level OpenGL functions.
\end{description}

\item The following optional libraries can be used as alternative to the
GtkGLArea widget of the GTK library to interact with OpenGL to draw graphs.
\begin{description}
\item[freeglut]\url{https://freeglut.sourceforge.net}
\item[sdl2]\url{https://www.libsdl.org}
\item[glfw]\url{https://www.glfw.org}
\end{description}

\item Optional to build the documentation:
\begin{description}
\item[doxygen]\url{https://www.doxygen.nl}
\item standard comments format to generate documentation.
\item[latex]\url{https://www.latex-project.org/}
\item to build the PDF manuals.
\end{description}

\end{itemize}

\subsection{Operative systems}

You can install all required utilities and libraries using the instructions of
\href{https://github.com/jburguete/install-unix}{install-unix}.

This software has been built and tested in the following operative systems:
\begin{itemize}
\item Arch Linux
\item Debian Linux 12
\item Devuan Linux 4
\item DragonFly BSD 6.4
\item Fedora Linux 38
\item FreeBSD 13.2
\item Gentoo Linux
\item Linux Mint DE 5
\item MacOS Ventura + Homebrew
\item Manjaro Linux
\item Microsoft Windows 10 64 bits + MSYS2
\item NetBSD 9.3
\item OpenBSD 7.3
\item OpenIndiana Hipster
\item OpenSUSE Linux Leap 15.5
\item Ubuntu Linux 23.04
\end{itemize}

On Microsoft Windows systems you have to install
\href{http://sourceforge.net/projects/msys2}{MSYS2} and the required
libraries and utilities. You can follow detailed instructions in
\href{https://github.com/jburguete/install-unix/blob/master/tutorial.pdf}{install-unix}
tutorial.

On NetBSD 9.3, to use the last GCC version, you have to do first on the
building terminal:
\begin{lstlisting}[language=bash]
$ export PATH="/usr/pkg/gcc12/bin:$PATH"
\end{lstlisting}
To do permanent this change the following line can be added to the ".profile"
file in the user root directory:
\begin{lstlisting}[language=bash]
$ PATH="/usr/pkg/gcc12/bin:$PATH"
\end{lstlisting}

On OpenBSD 7.2 you have to do first on the building terminal:
\begin{lstlisting}[language=bash]
$ export AUTOCONF_VERSION=2.69 AUTOMAKE_VERSION=1.16
\end{lstlisting}

\subsection{Building instructions}

Once installed and configurated all the tools and libraries, the sequence to
compile is made on a terminal:

\begin{enumerate}

\item Download the latest version of the
\href{https://github.com/jburguete/jb}{JB library}:
\begin{lstlisting}[language=bash]
$ git clone https://github.com/jburguete/jb.git
\end{lstlisting}

\item Build the JB library:
\begin{lstlisting}[language=bash]
$ cd jb/5.3.4
$ ./build.sh
$ cd ../..
\end{lstlisting}

\item Download this repository:
\begin{lstlisting}[language=bash]
$ git clone https://github.com/jburguete/surcos.git
\end{lstlisting}

\item Link the latest version of the JB library on the source directory to jb:
\begin{lstlisting}[language=bash]
$ cd surcos/6.0/src
$ ln -s ../../../jb/5.3.4 jb
$ cd ..
\end{lstlisting}

\item Build SURCOS doing on the terminal:
\begin{lstlisting}[language=bash]
$ ./build.sh
\end{lstlisting}

\item On Microsoft Windows you can do in order to create a framework to work
away of MSYS2:
\begin{lstlisting}[language=bash]
$ make dist
\end{lstlisting}

\end{enumerate}
