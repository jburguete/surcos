\chapter*{¡Atención!}

Para ejecutar el programa \emph{surcos} en sistema operativo Windows, deberá
iniciar la aplicación desde la subcarpeta \emph{win32/bin/winsurcos.exe}, en
versiones de 32 bits, o \emph{win64/bin/winsurcos.exe} en versiones de 64 bits.

\begin{itemize}
\item La representación de los números reales se hace, según el estándar
internacional, separando los decimales mediante el ``.'' decimal.
\item Las unidades de todas las variables utilizadas y representadas están en
Sistema Internacional.
\end{itemize}

El programa ha sido probado en los siguientes sistemas operativos de Microsoft:
\begin{itemize}
\item Windows 10 32 bits.
\item Windows 10 64 bits.
\end{itemize}

La activación del idioma se hace mediante la modificación de la configuración
regional del sistema operativo. La versión 5.6 de este programa tiene soporte
idiomático en inglés, español, francés e italiano. 

Se han reportado problemas en la ventana de representación gráfica debidos a
fallos de configuración de OpenGL en el driver de la tarjeta gráfica. Inténtese
en ese caso instalar el driver más actualizado de la tarjeta gráfica.

También se han reportado fallos guardando el fichero de la gráfica cuando el
sistema Windows se ejecuta dentro de una máquina virtual de VirtualBox. En este
caso el problema se corrige desactivando la aceleración gráfica en la
configuración de VirtualBox.

Windows 10 es marcas registradas de Microsoft.

\setcounter{page}{1}

\chapter{Instrucciones de instalación}

\section{Descarga}

El programa \emph{surcos} puede ser descargado libremente en:
\begin{itemize}
\item \textit{http://digital.csic.es/handle/10261/75830}
\end{itemize}
El código fuente más reciente puede descargarse libremente, con una licencia de
tipo BSD, en:
\begin{itemize}
\item \textit{https://github.com/jburguete/surcos}
\end{itemize}

\section{Instalación}

La instalación del programa \emph{surcos} consiste simplemente en descomprimirlo
en la carpeta deseada. No obstante, se recomienda que tanto la carpeta donde se
instala el programa como los nombres de los ficheros generados no contengan
espacios ni símbolos raros. Se han reportado casos en los que el podría no
encontrar los ficheros y no ejecutar la simulación.

\section{Ficheros del programa}

El programa consta de las siguientes carpetas:
\begin{description}
\item[win32/bin]
\item Carpeta que contiene el ejecutable, los ficheros ejecutables de las
librerías y los ficheros de diagramas en la versión para Windows de 32 bits.
\item[win32/etc]
\item[win32/lib]
\item Estas dos carpetas contienen algunos ficheros de las librerías para
Windows de 32 bits.
\item[win32/share]
\item Carpeta que contiene los ficheros de idiomas para Windows de 32 bits.
\item[win64/bin]
\item[win64/etc]
\item[win64/lib]
\item[win64/share]
\item Carpetas equivalentes a las anteriores para la versión de 64 bits.
\item[examples]
\item Carpeta que contiene ficheros de ejemplo.
\item[src]
\item Carpeta que contiene el código fuente del programa \emph{surcos} y el
código fuente de las librerías utilizadas.
\end{description}

\section{Compilación del código fuente}

El código fuente está escrito en lenguaje C y se han usado en su compilación
herramientas libres de GNU: gcc, gmake, aclocal, autoconf y pkg-config. La
versión para Windows se ha compilado además usando msys/mingw.

El programa \emph{surcos} hace uso de las siguientes librerías:
\begin{description}
\item[libiconv] (http://ftp.gnu.org/pub/gnu/libiconv)
\item[zlib] (http://sourceforge.net/projects/libpng)
\item[libxml] (http://xmlsoft.org)
\item[libffi] (ftp://sourceware.org/pub/libffi)
\item[glib] (http://ftp.gnome.org/pub/gnome/sources/glib)
\item[gettext] (http://ftp.gnu.org/pub/gnu/gettext)
\item[libpng] (http://sourceforge.net/projects/libpng)
\item[freetype] (http://sourceforge.net/projects/freetype)
\item[fontconfig] (http://fontconfig.freedesktop.org)
\item[pixman] (http://www.cairographics.org)
\item[cairo] (http://www.cairographics.org)
\item[atk] (http://ftp.gnome.org/pub/gnome/sources/atk)
\item[pango] (http://ftp.gnome.org/pub/gnome/sources/pango)
\item[gdk-pixbuf] (http://ftp.gnome.org/pub/gnome/sources/gdk-pixbuf)
\item[gtk+] (http://ftp.gnome.org/pub/gnome/sources/gtk+)
\item[freeglut] (http://sourceforge.net/projects/freeglut)
\item[jb] (https://github.com/jburguete/jb)
\end{description}

Una vez instaladas y configuradas todas estas herramientas y librerías la
secuencia para compilar el programa consiste en hacer en una terminal:
\begin{enumerate}
\item cd 5.6
\item ./build
\end{enumerate}
En algunos sistemas hay que hacer alguna corrección. Consúltese el inicio del
fichero README.md donde se dan instrucciones más detalladas en algunos sistemas.

El programa \emph{surcos} ha sido compilado y probado en los siguientes sistemas
operativos:
\begin{itemize}
\item Arch Linux
\item Debian Linux 10
\item Devuan Linux 3
\item DragonFly BSD 5.8
\item Dyson Illumos
\item Fedora Linux 32
\item FreeBSD 12.1
\item Linux Mint DE 3
\item Manjaro Linux
\item Microsoft Windows 10 32 bits
\item Microsoft Windows 10 64 bits
\item NetBSD 9.0
\item OpenBSD 6.7
\item OpenIndiana Hipster
\item OpenSUSE Linux Leap 15
\item Xubuntu Linux 20.04
\end{itemize}
