\chapter*{Warning!}

To execute the program \emph{surcos} on operative system Windows init the
application on the subdirectory \emph{win32/bin/winsurcos.exe}, with 32 bits
versions, or \emph{win64/bin/winsurcos.exe} with 64 bits versions.

\begin{itemize}
\item Real numbers are represented according to the international standard,
separating the integer and decimal parts by ``.''.
\item All units are in the International System.
\end{itemize}

The program has been tested in the following operative systems of Microsoft:
\begin{itemize}
\item Windows XP 32 bits.
\item Windows XP 64 bits.
\item Windows 7 32 bits.
\item Windows 7 64 bits.
\end{itemize}

The language activation is perfomed through the regional configuration in the 
operative system. Version 5.2 is offered in English, Spanish, French and Italian. 

Some problems with graphic representation have been reported due to OpenGL 
configuration failure in the graphical card driver. In that case, the latest driver
should be installed.

Other failures have been reported when saving the graphical file if Windows is
executed in a VirtualBox. In that case, the graphical acceleration should be desactivated
in the VirtualBox configuration.

Windows XP and Windows 7 are trademarks of Microsoft.

\setcounter{page}{1}

\chapter{Instalation instruction}

\section{Download}

Program \emph{surcos} can be freely downloaded from:
\begin{itemize}
\item \textit{http://digital.csic.es/handle/10261/75830}
\end{itemize}
The latest program \emph{surcos} source code can be freely downloaded, under
a BSD type licence, from:
\begin{itemize}
\item \textit{https://github.com/jburguete/surcos}
\end{itemize}

\section{Instalation}

To install \emph{surcos} it must only be decompressed in the desired folder.
It is recommended to avoid blank spaces and symbols in the names of the folder. 

\section{Program files}

The program contains the following folders:
\begin{description}
\item[win32/bin]
\item Folder with the executable file, the libraries executable files and the diagram files in the 
 Windows 32 bits version.
\item[win32/etc]
\item[win32/lib]
\item These two folders contain some files of the libraries for Windows 32 bits.
\item[win32/share]
\item Contains the language files for Windows 32 bits.
\item[win64/bin]
\item[win64/etc]
\item[win64/lib]
\item[win64/share]
\item Equivalent for Windows 64 bits.
\item[examples]
\item Example files.
\item[src]
\item Source code of \emph{surcos} and source code of the libraries used.
\end{description}

\section{Source code compilation}

The source code is written in C and has been compiled using the GNU free tools: gcc, gmake, 
aclocal, autoconf and pkg-config. The Windows version has been compiled using also msys/mingw.

Program \emph{surcos} uses the following libraries:
\begin{description}
\item[libiconv] (http://ftp.gnu.org/pub/gnu/libiconv)
\item[zlib] (http://sourceforge.net/projects/libpng)
\item[libxml] (http://xmlsoft.org)
\item[libffi] (ftp://sourceware.org/pub/libffi)
\item[glib] (http://ftp.gnome.org/pub/gnome/sources/glib)
\item[gettext] (http://ftp.gnu.org/pub/gnu/gettext)
\item[libpng] (http://sourceforge.net/projects/libpng)
\item[freetype] (http://sourceforge.net/projects/freetype)
\item[fontconfig] (http://fontconfig.freedesktop.org)
\item[pixman] (http://www.cairographics.org)
\item[cairo] (http://www.cairographics.org)
\item[atk] (http://ftp.gnome.org/pub/gnome/sources/atk)
\item[pango] (http://ftp.gnome.org/pub/gnome/sources/pango)
\item[gdk-pixbuf] (http://ftp.gnome.org/pub/gnome/sources/gdk-pixbuf)
\item[gtk+] (http://ftp.gnome.org/pub/gnome/sources/gtk+)
\item[freeglut] (http://sourceforge.net/projects/freeglut)
\end{description}

Once installed and configurated all the tools and libraries, the sequence to compile is made of four steps:

\begin{enumerate}
\item aclocal
\item autoconf
\item ./configure
\item make
\end{enumerate}
Some systems require corrections. Look at the beginning of the file configure.ac
where more detailed instructions are provided.  

The program \emph{surcos} has been compiled and tested in the following
operative systems:
\begin{itemize}
\item Debian Linux 7.4
\item Debian kFreeBSD 7.4
\item Debian Hurd 7.4
\item DragonFly BSD 3.2.2
\item FreeBSD 10.0
\item Microsoft Windows XP 32 bits
\item Microsoft Windows XP 64 bits
\item Microsoft Windows 7 32 bits
\item Microsoft Windows 7 64 bits
\item NetBSD 6.1
\item OpenBSD 5.2
\item OpenIndiana 151a7
\end{itemize}
